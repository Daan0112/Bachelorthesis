\documentclass{article}
\usepackage[utf8]{inputenc}
\usepackage{geometry}
\usepackage{graphicx}
\usepackage{booktabs}
\usepackage{hyperref}

\title{EDA and ABM Assignment: Deliverables}
\author{[Your Name]}
\date{\today}

\begin{document}

\maketitle

\section{EDA}

\subsection{Task 1: Sanity Checks \& Structure}

\textit{Goal: Describe the dataset subjects, time points, and data quality (e.g., negative values, zeros).}

\subsubsection{Dataset Description}
% Instruction: Write a short paragraph (5–10 lines) describing:
% - Number of subjects
% - Existing time points
% - Any obvious issues (e.g., subsets always zero, missing values)

[Insert Dataset Description Text Here]

\subsubsection{Summary Table}
% Instruction: Include a small table summarizing the dataset properties (see document/notebook).

\begin{table}[h!]
    \centering
    \begin{tabular}{ccc}
        \toprule
        Column 1 & Column 2 & Column 3 \\
        \midrule
        Data & Data & Data \\
        Data & Data & Data \\
        \bottomrule
    \end{tabular}
    \caption{[Insert Table Caption]}
    \label{tab:dataset_summary}
\end{table}

\subsection{Task 2: Time-course Summary}

\textit{Goal: Describe the typical CD4 response over time using median curves.}

\subsubsection{Plots: Median $\pm$ IQR}
% Instruction:
% - Plot 1: Median values for Tnaive, TSCM, TCM, TEMRA vs Time (0–60 days). Optional: Shaded IQR.
% - Plot 2: (Optional) Medians at long-term (~356/365 days).

\begin{figure}[h!]
    \centering
    % \includegraphics[width=0.8\textwidth]{path/to/plot1.png}
    \caption{[Median CD4 T-cell Subset Responses Over Time]}
    \label{fig:time_course_plot}
\end{figure}

[Insert Plots Here]

\subsubsection{Key Time Points Table}
% Instruction: A summary table with key time points.

\begin{table}[h!]
    \centering
    \begin{tabular}{ccc}
        \toprule
        Time Point & Subset & Median Value \\
        \midrule
        Day 1 & ... & ... \\
        Day 22 & ... & ... \\
        Day 43 & ... & ... \\
        Day 365 & ... & ... \\
        \bottomrule
    \end{tabular}
    \caption{[Key Time Points Summary]}
    \label{tab:time_points}
\end{table}

\section{ABM}

\subsection{Model Implementation}

\textit{Goal: Build a day-by-day model of the CD4 response.}

\subsubsection{Model Description}
% Instruction: Define the model and parameters. The python notebook should contain the code.

[Insert Model Description / Reference to Python Code Here]

\subsection{Simulation Results}

\textit{Goal: reproduce the shape of the median curves from the EDA.}

\subsubsection{Model Trajectories}
% Instruction: One figure showing model trajectories for S, C, R (and optionally N) versus time (0 to 365 days).

\begin{figure}[h!]
    \centering
    % \includegraphics[width=0.8\textwidth]{path/to/model_trajectories.png}
    \caption{[Model Trajectories: S, C, R vs Time]}
    \label{fig:model_trajectories}
\end{figure}

[Insert Model Trajectories Figure Here]

\subsubsection{Model vs Data Comparison}
% Instruction: Figures comparing model vs data medians:
% - S vs TSCM
% - C vs TCM
% - R vs TEMRA
% - Total Memory (S+C+R) vs Total CD4 Data

\begin{figure}[h!]
    \centering
    % \includegraphics[width=0.8\textwidth]{path/to/comparison_plot.png}
    \caption{[Model vs Data Comparison]}
    \label{fig:model_vs_data}
\end{figure}

[Insert Comparison Figures Here]

\subsection{Discussion}

\textit{Goal: Explain parameters and model performance.}

\subsubsection{Final Parameters}
% Instruction: List the final parameter values used (T_antigen, p_act, frac_memory, p_SC, p_CR, d_S, d_C, d_R).

[Insert Parameter List Here]

\subsubsection{Explanation}
% Instruction: Short explanation (1/2 – 1 page) covering:
% - How well the model reproduces CD4 memory persistence.
% - What it gets right/wrong.
% - Effect of changing frac_memory (memory-dominant vs TEMRA-skewed).

[Insert Explanation Text Here]

\end{document}
